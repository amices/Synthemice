%  LaTeX support: latex@mdpi.com
%  In case you need support, please attach all files that are necessary for compiling as well as the log file, and specify the details of your LaTeX setup (which operating system and LaTeX version / tools you are using).

%=================================================================
\documentclass[psych,article,submit,moreauthors,pdftex]{mdpi}

% If you would like to post an early version of this manuscript as a preprint, you may use preprint as the journal and change 'submit' to 'accept'. The document class line would be, e.g., \documentclass[preprints,article,accept,moreauthors,pdftex]{mdpi}. This is especially recommended for submission to arXiv, where line numbers should be removed before posting. For preprints.org, the editorial staff will make this change immediately prior to posting.

%% Some pieces required from the pandoc template
\providecommand{\tightlist}{%
  \setlength{\itemsep}{0pt}\setlength{\parskip}{4pt}}
\setlist[itemize]{leftmargin=*,labelsep=5.8mm}
\setlist[enumerate]{leftmargin=*,labelsep=4.9mm}

\usepackage{longtable}

% see https://stackoverflow.com/a/47122900
\usepackage{color}
\usepackage{fancyvrb}
\newcommand{\VerbBar}{|}
\newcommand{\VERB}{\Verb[commandchars=\\\{\}]}
\DefineVerbatimEnvironment{Highlighting}{Verbatim}{commandchars=\\\{\}}
% Add ',fontsize=\small' for more characters per line
\usepackage{framed}
\definecolor{shadecolor}{RGB}{248,248,248}
\newenvironment{Shaded}{\begin{snugshade}}{\end{snugshade}}
\newcommand{\AlertTok}[1]{\textcolor[rgb]{0.94,0.16,0.16}{#1}}
\newcommand{\AnnotationTok}[1]{\textcolor[rgb]{0.56,0.35,0.01}{\textbf{\textit{#1}}}}
\newcommand{\AttributeTok}[1]{\textcolor[rgb]{0.77,0.63,0.00}{#1}}
\newcommand{\BaseNTok}[1]{\textcolor[rgb]{0.00,0.00,0.81}{#1}}
\newcommand{\BuiltInTok}[1]{#1}
\newcommand{\CharTok}[1]{\textcolor[rgb]{0.31,0.60,0.02}{#1}}
\newcommand{\CommentTok}[1]{\textcolor[rgb]{0.56,0.35,0.01}{\textit{#1}}}
\newcommand{\CommentVarTok}[1]{\textcolor[rgb]{0.56,0.35,0.01}{\textbf{\textit{#1}}}}
\newcommand{\ConstantTok}[1]{\textcolor[rgb]{0.00,0.00,0.00}{#1}}
\newcommand{\ControlFlowTok}[1]{\textcolor[rgb]{0.13,0.29,0.53}{\textbf{#1}}}
\newcommand{\DataTypeTok}[1]{\textcolor[rgb]{0.13,0.29,0.53}{#1}}
\newcommand{\DecValTok}[1]{\textcolor[rgb]{0.00,0.00,0.81}{#1}}
\newcommand{\DocumentationTok}[1]{\textcolor[rgb]{0.56,0.35,0.01}{\textbf{\textit{#1}}}}
\newcommand{\ErrorTok}[1]{\textcolor[rgb]{0.64,0.00,0.00}{\textbf{#1}}}
\newcommand{\ExtensionTok}[1]{#1}
\newcommand{\FloatTok}[1]{\textcolor[rgb]{0.00,0.00,0.81}{#1}}
\newcommand{\FunctionTok}[1]{\textcolor[rgb]{0.00,0.00,0.00}{#1}}
\newcommand{\ImportTok}[1]{#1}
\newcommand{\InformationTok}[1]{\textcolor[rgb]{0.56,0.35,0.01}{\textbf{\textit{#1}}}}
\newcommand{\KeywordTok}[1]{\textcolor[rgb]{0.13,0.29,0.53}{\textbf{#1}}}
\newcommand{\NormalTok}[1]{#1}
\newcommand{\OperatorTok}[1]{\textcolor[rgb]{0.81,0.36,0.00}{\textbf{#1}}}
\newcommand{\OtherTok}[1]{\textcolor[rgb]{0.56,0.35,0.01}{#1}}
\newcommand{\PreprocessorTok}[1]{\textcolor[rgb]{0.56,0.35,0.01}{\textit{#1}}}
\newcommand{\RegionMarkerTok}[1]{#1}
\newcommand{\SpecialCharTok}[1]{\textcolor[rgb]{0.00,0.00,0.00}{#1}}
\newcommand{\SpecialStringTok}[1]{\textcolor[rgb]{0.31,0.60,0.02}{#1}}
\newcommand{\StringTok}[1]{\textcolor[rgb]{0.31,0.60,0.02}{#1}}
\newcommand{\VariableTok}[1]{\textcolor[rgb]{0.00,0.00,0.00}{#1}}
\newcommand{\VerbatimStringTok}[1]{\textcolor[rgb]{0.31,0.60,0.02}{#1}}
\newcommand{\WarningTok}[1]{\textcolor[rgb]{0.56,0.35,0.01}{\textbf{\textit{#1}}}}

%--------------------
% Class Options:
%--------------------
%----------
% journal
%----------
% Choose between the following MDPI journals:
% acoustics, actuators, addictions, admsci, aerospace, agriculture, agriengineering, agronomy, algorithms, animals, antibiotics, antibodies, antioxidants, applsci, arts, asc, asi, atmosphere, atoms, axioms, batteries, bdcc, behavsci , beverages, bioengineering, biology, biomedicines, biomimetics, biomolecules, biosensors, brainsci , buildings, cancers, carbon , catalysts, cells, ceramics, challenges, chemengineering, chemistry, chemosensors, children, cleantechnol, climate, clockssleep, cmd, coatings, colloids, computation, computers, condensedmatter, cosmetics, cryptography, crystals, dairy, data, dentistry, designs , diagnostics, diseases, diversity, drones, econometrics, economies, education, electrochem, electronics, energies, entropy, environments, epigenomes, est, fermentation, fibers, fire, fishes, fluids, foods, forecasting, forests, fractalfract, futureinternet, futurephys, galaxies, games, gastrointestdisord, gels, genealogy, genes, geohazards, geosciences, geriatrics, hazardousmatters, healthcare, heritage, highthroughput, horticulturae, humanities, hydrology, ijerph, ijfs, ijgi, ijms, ijns, ijtpp, informatics, information, infrastructures, inorganics, insects, instruments, inventions, iot, j, jcdd, jcm, jcp, jcs, jdb, jfb, jfmk, jimaging, jintelligence, jlpea, jmmp, jmse, jnt, jof, joitmc, jpm, jrfm, jsan, land, languages, laws, life, literature, logistics, lubricants, machines, magnetochemistry, make, marinedrugs, materials, mathematics, mca, medicina, medicines, medsci, membranes, metabolites, metals, microarrays, micromachines, microorganisms, minerals, modelling, molbank, molecules, mps, mti, nanomaterials, ncrna, neuroglia, nitrogen, notspecified, nutrients, ohbm, particles, pathogens, pharmaceuticals, pharmaceutics, pharmacy, philosophies, photonics, physics, plants, plasma, polymers, polysaccharides, preprints , proceedings, processes, proteomes, psych, publications, quantumrep, quaternary, qubs, reactions, recycling, religions, remotesensing, reports, resources, risks, robotics, safety, sci, scipharm, sensors, separations, sexes, signals, sinusitis, smartcities, sna, societies, socsci, soilsystems, sports, standards, stats, surfaces, surgeries, sustainability, symmetry, systems, technologies, test, toxics, toxins, tropicalmed, universe, urbansci, vaccines, vehicles, vetsci, vibration, viruses, vision, water, wem, wevj

%---------
% article
%---------
% The default type of manuscript is "article", but can be replaced by:
% abstract, addendum, article, benchmark, book, bookreview, briefreport, casereport, changes, comment, commentary, communication, conceptpaper, conferenceproceedings, correction, conferencereport, expressionofconcern, extendedabstract, meetingreport, creative, datadescriptor, discussion, editorial, essay, erratum, hypothesis, interestingimages, letter, meetingreport, newbookreceived, obituary, opinion, projectreport, reply, retraction, review, perspective, protocol, shortnote, supfile, technicalnote, viewpoint
% supfile = supplementary materials

%----------
% submit
%----------
% The class option "submit" will be changed to "accept" by the Editorial Office when the paper is accepted. This will only make changes to the frontpage (e.g., the logo of the journal will get visible), the headings, and the copyright information. Also, line numbering will be removed. Journal info and pagination for accepted papers will also be assigned by the Editorial Office.

%------------------
% moreauthors
%------------------
% If there is only one author the class option oneauthor should be used. Otherwise use the class option moreauthors.

%---------
% pdftex
%---------
% The option pdftex is for use with pdfLaTeX. If eps figures are used, remove the option pdftex and use LaTeX and dvi2pdf.

%=================================================================
\firstpage{1}
\makeatletter
\setcounter{page}{\@firstpage}
\makeatother
\pubvolume{xx}
\issuenum{1}
\articlenumber{5}
\pubyear{2019}
\copyrightyear{2019}
%\externaleditor{Academic Editor: name}
\history{Received: date; Accepted: date; Published: date}
\updates{yes} % If there is an update available, un-comment this line

%% MDPI internal command: uncomment if new journal that already uses continuous page numbers
%\continuouspages{yes}

%------------------------------------------------------------------
% The following line should be uncommented if the LaTeX file is uploaded to arXiv.org
%\pdfoutput=1

%=================================================================
% Add packages and commands here. The following packages are loaded in our class file: fontenc, calc, indentfirst, fancyhdr, graphicx, lastpage, ifthen, lineno, float, amsmath, setspace, enumitem, mathpazo, booktabs, titlesec, etoolbox, amsthm, hyphenat, natbib, hyperref, footmisc, geometry, caption, url, mdframed, tabto, soul, multirow, microtype, tikz

%=================================================================
%% Please use the following mathematics environments: Theorem, Lemma, Corollary, Proposition, Characterization, Property, Problem, Example, ExamplesandDefinitions, Hypothesis, Remark, Definition
%% For proofs, please use the proof environment (the amsthm package is loaded by the MDPI class).

%=================================================================
% Full title of the paper (Capitalized)
\Title{Anony\emph{mice}d shareable data: Using \emph{mice} to create and
analyze multiply imputed synthetic data sets}

% Authors, for the paper (add full first names)
\Author{Thom Volker$^{1,*,
\dagger}$\href{https://orcid.org/0000-0002-2408-7820}{\orcidicon}, Gerko
Vink$^{1,
\dagger}$\href{https://orcid.org/0000-0001-9767-1924}{\orcidicon}}

% Authors, for metadata in PDF
\AuthorNames{Thom Volker, Gerko Vink}

% Affiliations / Addresses (Add [1] after \address if there is only one affiliation.)
\address{%
$^{1}$ \quad Utrecht University - Department of Methodology and
Statistics. Padualaan 14. 3584CH Utrecht, the Netherlands; \\
}
% Contact information of the corresponding author
\corres{Correspondence: \href{mailto:t.b.volker@uu.nl}{\nolinkurl{t.b.volker@uu.nl}}.}

% Current address and/or shared authorship
\firstnote{These authors contributed equally to this work.}







% The commands \thirdnote{} till \eighthnote{} are available for further notes

% Simple summary

% Abstract (Do not insert blank lines, i.e. \\)
\abstract{Synthetic data sets can greatly improve the dissemination of
data and further analysis of private data. Generating and analyzing
synthetic data sets is straightforward, yet a synthetic data analysis
pipeline is seldom adopted by applied researchers. We outline a simple
procedure for generating and analyzing synthetic data sets with
\texttt{mice} in \texttt{R}. We demonstrate using simulations that the
analysis results obtained on synthetic data yields unbiased and valid
inferences, and leads to synthetic records that cannot be distinguished
from the true data records. The ease of use when synthesizing data with
\texttt{mice}, together with the validity of inferences obtained through
this procedure opens up a wealth of possibilities for data dissemination
and further research of initially private data.}

% Keywords
\keyword{mice; imputation; synthetic data.}

% The fields PACS, MSC, and JEL may be left empty or commented out if not applicable
%\PACS{J0101}
%\MSC{}
%\JEL{}

%%%%%%%%%%%%%%%%%%%%%%%%%%%%%%%%%%%%%%%%%%
% Only for the journal Diversity
%\LSID{\url{http://}}

%%%%%%%%%%%%%%%%%%%%%%%%%%%%%%%%%%%%%%%%%%
% Only for the journal Applied Sciences:
%\featuredapplication{Authors are encouraged to provide a concise description of the specific application or a potential application of the work. This section is not mandatory.}
%%%%%%%%%%%%%%%%%%%%%%%%%%%%%%%%%%%%%%%%%%

%%%%%%%%%%%%%%%%%%%%%%%%%%%%%%%%%%%%%%%%%%
% Only for the journal Data:
%\dataset{DOI number or link to the deposited data set in cases where the data set is published or set to be published separately. If the data set is submitted and will be published as a supplement to this paper in the journal Data, this field will be filled by the editors of the journal. In this case, please make sure to submit the data set as a supplement when entering your manuscript into our manuscript editorial system.}

%\datasetlicense{license under which the data set is made available (CC0, CC-BY, CC-BY-SA, CC-BY-NC, etc.)}

%%%%%%%%%%%%%%%%%%%%%%%%%%%%%%%%%%%%%%%%%%
% Only for the journal Toxins
%\keycontribution{The breakthroughs or highlights of the manuscript. Authors can write one or two sentences to describe the most important part of the paper.}

%\setcounter{secnumdepth}{4}
%%%%%%%%%%%%%%%%%%%%%%%%%%%%%%%%%%%%%%%%%%

% Pandoc citation processing

\usepackage{booktabs}
\usepackage{longtable}
\usepackage{array}
\usepackage{multirow}
\usepackage{wrapfig}
\usepackage{float}
\usepackage{colortbl}
\usepackage{pdflscape}
\usepackage{tabu}
\usepackage{threeparttable}
\usepackage{threeparttablex}
\usepackage[normalem]{ulem}
\usepackage{makecell}
\usepackage{xcolor}

\begin{document}
%%%%%%%%%%%%%%%%%%%%%%%%%%%%%%%%%%%%%%%%%%

\hypertarget{introduction}{%
\section{Introduction}\label{introduction}}

Open science, including open data, has been marked as the future of
science \citep{gewin_data_2016}, and the advantages of publicly
available research data are numerous
\citep{molloy_open_2011, walport_brest_sharing_2011}. Collecting
research data requires an enormous investment both in terms of time and
monetary resources. Openly accessible research data bears the potential
of increasing the scientific returns for the same data collection
effort. Additionally, the fact that public funds are used for data
collection results in increasing demand for access to the collected
data. Nevertheless, the possibilities to distribute research data
directly are often very limited due to restrictions on data privacy and
data confidentiality. Although these regulations are much needed,
privacy constraints are also ranked among the toughest challenges to
overcome in the advancement of modern day social science research
\citep{lazer_life_2009}.

Anonymizing research data might seem a quick and appealing approach to
limit the unique identification of participants. However, this approach
is not sufficient to fulfil contemporary privacy and confidentiality
requirements \citep{ohm_broken_2009, national_putting_2007}. Over the
years, several other techniques have been used to increase the
confidentiality of research data, such as categorizing continuous
variables, top coding values above an upper bound or adding random noise
to the observed values \citep{drechsler_synthetic_2011}. However, these
methods may distort the true data relation between variables, thereby
reducing the data quality and the scientific returns for re-using the
same data for further research.

An alternative solution has been proposed separately by
\citet{rubin_statistical_disclosure_1993} and
\citet{little_statistical_1993}. Although their approaches differ to
some extent, the overarching procedure is to use bonafide observed data
to generate multiply imputed synthetic data sets that can be freely
disclosed. While in practice, one could see this as replacing the
observed data values by multiple draws from the posterior predictive
distribution of the observed data, based on some imputation model, Rubin
would argue that these synthetic data values are merely draws from the
same true data generating model. In that sense, the observed data is
never replaced, but the population is resampled from the information
captured in the (incomplete) sample. Using this approach, the researcher
could replace the observed data set as a whole with multiple synthetic
versions. Alternatively, the researcher could opt to only replace a
subset of the observed data. For example, one can choose to only replace
dimensions in the data that could be compared with publicly available
data sets or registers. Likewise, synthetisation could be limited to
those values that are disclosive, such as high incomes or high
turnovers.

Conceptually, the synthetic data framework is based upon the building
blocks of multiple imputation of missing data, as proposed by
\citet{rubin_multiple_1987}. Instead of replacing just the missing
values with multiple draws from the posterior predictive distribution,
one could easily \emph{overimpute} any observed sensitive values.
Similarly to multiple imputation of missing data, the multiple synthetic
data sets allow for correct statistical inferences, despite the fact
that the analyses do not use the ``true'' value. The analyses over
multiple synthetic data sets should be pooled into a single inference,
so that the researcher can draw valid conclusions from the pooled
results. To that respect, the variance should reflect the added
variability that is induced by the imputation procedure.

Potentially, this approach could fulfill the needs for openly accessibly
data, without running into barriers with regard to privacy and
confidentiality constraints. However, there is no such thing as a free
lunch: data collectors have to put effort in creating high-quality
synthetic data. The quality of the synthetic data is highly dependent on
the imputation models, and using flawed models to generate synthetic
data might bias subsequent analyses
\citep{reiter2004simultaneous, grund2021using, jiang2021balancing}.
Conversely, if the models used to create the synthetic data are able to
preserve the relationships between the variables as in the original
data, the synthetic data can be nearly as informative as the observed
data. Thus, to fully exploit the benefits of synthetic data, additional
complications to actually create these high-quality data sets should be
kept at a minimum.

To mitigate unnecessary challenges related to creating synthetic data
sets on behalf of the researcher, software aimed at multiple imputation
of missing data can be employed. Especially if researchers acquired
familiarity with this software during earlier projects, or used it
earlier during the research process, the additional burden of creating
synthetic data sets is relatively small. The R-package \texttt{mice}
\citep{mice} implements multiple imputation of missing data in a
straightforward and user-friendly manner. However, the functionality of
\texttt{mice} is not restricted to the imputation of missing data, but
allows to impute any value in the data: even observed values.
Consequently, \texttt{mice} can be utilized for the creation of multiply
imputed synthetic data sets.

After creating multiply imputed synthetic data sets, the goal is to
obtain valid statistical inferences in the spirit of
\citet{rubin_multiple_1987} and \citet{neyman1934}. In the missing data
framework, this is done by performing statistical analyses on all
imputed data sets, and pooling the results of the analyses according to
Rubin's rules \citep[p.~76]{rubin_multiple_1987}. In the synthetic data
framework, the same procedure is followed, but with a slight twist:
there are no values that remain constant over the synthetic data sets.
The procedure of drawing valid inferences from multiple synthetic data
sets is therefore slightly different.

In this manuscript we detail a workflow for synthesizing data with
\texttt{mice}. First, the \texttt{mice} algorithm for the creation of
synthetic data will be shortly explained. The aim is to generate
synthetic sets that reassure the privacy and confidentiality of the
participants. Second, a straightforward workflow for imputation of
synthetic data with \texttt{mice} will be demonstrated. Third, we
demonstrate the validity of the procedure through statistical
simulation.

\hypertarget{generating-synthetic-data-with-mice}{%
\section{\texorpdfstring{Generating synthetic data with
\texttt{mice}}{Generating synthetic data with mice}}\label{generating-synthetic-data-with-mice}}

The \texttt{mice} package \citep{mice} in \texttt{R} \citep{Rproject}
has been developed for multiple imputation to overcome problems related
to nonresponse. In that context, the aim is to replace missing values
with plausible values from the posterior predictive distribution of that
variable. In \texttt{mice}, this is accomplished using fully conditional
specification (FCS) \citep{vanbuuren_fully_2006}, which breaks down the
multivariate distribution of the data
\(\textbf{Y} = (\textbf{Y}_{obs}, \textbf{Y}_{mis})\) into
\(j = 1, 2, \dots, k\) univariate conditional densities, where \(k\)
denotes the number of columns in the data. Using FCS, a model is
constructed for every incomplete variable and the missing values
\(Y_{j, mis}\) are then imputed with draws from the posterior predictive
distribution of \(P(Y_{j, mis} | \textbf{Y}_{obs}, \theta)\) on a
variable-by-variable basis. Note that the predictor matrix \(Y_{-j}\)
may contain yet imputed values from an earlier imputation step, and thus
will be updated after every iteration. This procedure is applied \(m\)
times, resulting in \(m\) completed data sets
\(\textbf{D} = (\textbf{D}^{(1)}, \textbf{D}^{(2)}, \dots, \textbf{D}^{(m)})\),
with \(\textbf{D}^{(l)} = (\textbf{Y}_{obs}, Y^{(l)}_{mis})\).

In \texttt{mice}, the generation of multiply imputed data sets to solve
for unobserved values is straightforward. The following pseudocode
details the multiple imputation of the \texttt{mice::boys} data set
\citep{fredriks_boys_2000} into the object \texttt{imp} with
\texttt{m\ =\ 10} imputated sets and \texttt{maxit\ =\ 7} iterations for
the algorithm to converge, using the default imputations methods for
each column data class.

\begin{Shaded}
\begin{Highlighting}[]
\FunctionTok{library}\NormalTok{(mice)}
\NormalTok{imp }\OtherTok{\textless{}{-}} \FunctionTok{mice}\NormalTok{(boys, }
            \AttributeTok{m =} \DecValTok{10}\NormalTok{,}
            \AttributeTok{maxit =} \DecValTok{7}\NormalTok{)}
\end{Highlighting}
\end{Shaded}

It is straightforward to extend the imputation approach to generate
synthetic values. Rather than imputing missing data, the observed values
are then replaced by synthetic draws from the posterior predictive
distribution. For simplicity, assume that the data is completely
observed (i.e., \(\textbf{Y} = \textbf{Y}_{obs}\)). Following the
notation of \citet{reiter_raghunathan_multiple_2007}, let for \(n\)
units denote \(Z_i = 1\) if any of the values of unit
\(i = 1, 2, \dots, n\) are to be replaced by imputations, and
\(Z_i = 0\) otherwise, with \(\textbf{Z} = (Z_1, Z_2, \dots, Z_n)\).
Accordingly, the data consists of values that are to be replaced and
values that are to be kept (i.e.,
\(\textbf{Y} = (\textbf{Y}_{rep}, \textbf{Y}_{nrep})\). Now, instead of
imputing \(\textbf{Y}_{mis}\) with draws from the posterior predictive
distribution of \(P(Y_{j, mis} | \textbf{Y}_{obs}, \theta)\) as in the
missing data case, \(\textbf{Y}_{rep}\) is imputed from the posterior
distribution of
\(P(Y^{(l)}_{j, rep} | \textbf{Y}^{(l)}_{-j}, \textbf{Z}, \theta)\),
where \(l\) is an indicator for the synthetic data set
(\(l = 1, 2, \dots, m\)). Note that synthetic values that are imputed at
an earlier step can be used for imputing variable \(j\). This process
results in the synthetic data
\(\textbf{D} = (\textbf{D}^{(1)}, \textbf{D}^{(2)}, \dots, \textbf{D}^{(m)})\).

For example, overimputing synthetic values for both the observed and
missing cells in the \texttt{mice::boys} data set into the object
\texttt{syn}, given the same imputation parameters as before, can be
realized by the following code execution.

\begin{Shaded}
\begin{Highlighting}[]
\NormalTok{syn }\OtherTok{\textless{}{-}} \FunctionTok{mice}\NormalTok{(boys, }
            \AttributeTok{m =} \DecValTok{10}\NormalTok{,}
            \AttributeTok{maxit =} \DecValTok{7}\NormalTok{, }
            \AttributeTok{where =} \FunctionTok{matrix}\NormalTok{(}\ConstantTok{TRUE}\NormalTok{, }
                           \AttributeTok{nrow =} \FunctionTok{nrow}\NormalTok{(boys),}
                           \AttributeTok{ncol =} \FunctionTok{ncol}\NormalTok{(boys)))}
\end{Highlighting}
\end{Shaded}

where the argument \texttt{where} requires a matrix of the same
dimensions as the data, (i.e., a \(n \times k\) matrix) containing
logicals \(z_{ij}\) that indicate which cells are selected to have their
values replaced by draws from the posterior predictive distribution.
This approach allows to overimpute a subset of the observed data, or as
in the above example, the observed data as a whole, resulting in a data
set that partially or completely consists of synthetic data values.

Choosing an adequate imputation model to impute the data is paramount,
as a flawed imputation model may drastically impact the validity of
inferences \citep{grund2021using, jiang2021balancing}. Imputation models
should be as flexible as possible to capture most of the patterns in the
data, and to model possibly unanticipated data characteristics
\citep{murray_multiple_2018, rubin_18years_1996}. Parametric methods,
albeit easy to implement in practice, may be too restrictive to capture
generally complex patterns in the data, especially in the case of
nonlinear relations and interactions between multiple variables.
Classification and regression trees \citep[CART;][]{breiman_cart_1984}
allow to model more complex patterns in the data, and have therefore
been suggested as an appropriate imputation method
\citep{reiter_cart_2005, burgette_reiter_cart_2010, doove_buuren_recursive_2014}.
Loosely speaking, CART sequentially splits the predictor space into
non-overlapping regions in such a way that the within-region variance is
as small as possible after every split. As such, CART does not impose
any parametric distribution on the data, making it a widely applicable
method that allows for a large variety of relationships within the data
\citep{islr_2013}. Given these appealing characteristics and the call
for the use of flexible methods when multiply imputing data, we will
focus our illustrations and evaluations of \texttt{mice} to method
\texttt{mice.impute.cart()}, realized by:

\begin{Shaded}
\begin{Highlighting}[]
\NormalTok{syn }\OtherTok{\textless{}{-}} \FunctionTok{mice}\NormalTok{(boys, }
            \AttributeTok{m =} \DecValTok{10}\NormalTok{,}
            \AttributeTok{maxit =} \DecValTok{7}\NormalTok{, }
            \AttributeTok{method =} \StringTok{"cart"}\NormalTok{,}
            \AttributeTok{where =} \FunctionTok{matrix}\NormalTok{(}\ConstantTok{TRUE}\NormalTok{, }
                           \AttributeTok{nrow =} \FunctionTok{nrow}\NormalTok{(boys),}
                           \AttributeTok{ncol =} \FunctionTok{ncol}\NormalTok{(boys)))}
\end{Highlighting}
\end{Shaded}

In a nutshell, the above code shows the simplicity of creating
\(m = 10\) synthetic data sets using \texttt{mice}. In practice,
however, one should take some additional complicating factors into
account. For example, one should account for deterministic relations in
the data. Additionally, relations between variables may be described
best using a different model than \texttt{CART}. Such factors are data
dependent and should be considered by the imputer. In the next section,
we will describe how the \texttt{boys} data can be adequately imputed.
Additionally, we will show through simulations that this approach yields
valid inferences.

\hypertarget{materials-and-methods}{%
\section{Materials and Methods}\label{materials-and-methods}}

We demonstrate the suitability of using \texttt{mice} for synthetisation
using a simulation study on the \texttt{mice::boys} data set. This data
set consists of the values of \(748\) Dutch boys on the following \(9\)
variables:

\begin{table}[H]
\caption{Description of the features in the \texttt{mice::boys} data set.}
\centering
\begin{tabular}{rl}
\hline
column & description                \\ 
\hline
age    & age in years               \\ 
hgt    & height (cm)                \\ 
wgt    & weight (kg)                \\ 
bmi    & body mass index            \\ 
hc     & head circumference (cm)    \\ 
gen    & genital Tanner stage G1-G5 \\ 
phb    & pubic hair Tanner P1-P6    \\ 
tv     & testicular volume (ml)     \\ 
reg    & region                     \\ 
\hline
\end{tabular}
\end{table}

Unfortunately, this data set does not differ from the vast majority of
collected data sets, in the sense that it suffers from missing data. For
simplicity, we complete the missing values using the default
\texttt{mice} imputation model for all predictors except \texttt{bmi},
which is passively imputed using its deterministic relation with
\texttt{wgt} and \texttt{hgt}. Specifically, the imputed values are used
to calculate the exact \texttt{bmi} values that correspond with
\texttt{hgt} and \texttt{wgt}.

\begin{Shaded}
\begin{Highlighting}[]
\CommentTok{\# create a single imputed, completely observed \textasciigrave{}boys\textasciigrave{} data set}
\FunctionTok{set.seed}\NormalTok{(}\DecValTok{123}\NormalTok{)}

\NormalTok{meth }\OtherTok{\textless{}{-}} \FunctionTok{make.method}\NormalTok{(boys)}
\NormalTok{meth[}\StringTok{"bmi"}\NormalTok{] }\OtherTok{\textless{}{-}} \StringTok{"\textasciitilde{} I(wgt / (hgt / 100)\^{}2)"}
\NormalTok{pred }\OtherTok{\textless{}{-}} \FunctionTok{make.predictorMatrix}\NormalTok{(boys)}
\NormalTok{pred[}\FunctionTok{c}\NormalTok{(}\StringTok{"hgt"}\NormalTok{, }\StringTok{"wgt"}\NormalTok{), }\StringTok{"bmi"}\NormalTok{] }\OtherTok{\textless{}{-}} \DecValTok{0}

\NormalTok{imp }\OtherTok{\textless{}{-}} \FunctionTok{mice}\NormalTok{(boys, }
            \AttributeTok{m =} \DecValTok{1}\NormalTok{,}
            \AttributeTok{maxit =} \DecValTok{10}\NormalTok{,}
            \AttributeTok{method =}\NormalTok{ meth,}
            \AttributeTok{predictorMatrix =}\NormalTok{ pred)}

\NormalTok{data }\OtherTok{\textless{}{-}} \FunctionTok{complete}\NormalTok{(imp)}
\end{Highlighting}
\end{Shaded}

\hypertarget{simulation-methods}{%
\subsection{Simulation methods}\label{simulation-methods}}

Usually, one would draw samples from a population that can be
synthesized to evaluate the performance of the synthesization methods.
As we only have access to a single sample, 1000 bootstrap samples of the
\texttt{boys} data have been synthesized with \(m = 5\) imputations for
every data cell to induce an appropriate amount of sampling variance.
Synthetic values are generated using the \texttt{CART} imputation method
for all columns, except for \texttt{bmi}. The deterministic relation
\texttt{bmi} which will be synthesized passively based on the synthetic
values for \texttt{hgt} and \texttt{wgt} to preserve the relation in the
synthetic data. Additional parameters that come with the use of
\texttt{mice.impute.cart()} are the complexity parameter \texttt{cp} and
the minimum number of observations in any terminal node
\texttt{minbucket}, that both constrain the flexibility of the
imputation model. The values of the parameters \texttt{cp} and
\texttt{minbucket} ought to adhere to the call for imputation models
that are as flexible as possible. Appropriate values for these
parameters, as well as the input for the \texttt{predictorMatrix},
depend on the data at hand. In the current example, the complexity
parameter is specified at \texttt{cp\ =\ 1e-08} rather than the default
value \texttt{1e-04}, and the minimum number of observations in each
terminal node is set at \texttt{minbucket} \(= 3\) rather than the
default value \(5\). By allowing for more complexity in the imputation
model, bias in the estimates from the synthetic data set is reduced.
Additionally, since the synthesis pattern is monotone, the number of
iterations can be set to \texttt{maxit\ =\ 1} \citep[e.g.,][Ch.
3]{drechsler_synthetic_2011}.

To assess the performance of \texttt{mice} for synthesizing data, we
compare the bootstrapped samples with the synthetic versions of these
bootstrapped samples. Specifically, univariate descriptive statistics,
the correlation matrix, and two linear regression models as well as one
ordered logistic regression model will be considered. Subsequently, the
bias in the parameters and the \(95\%\) confidence interval coverage of
the synthetic data will be examined. Similar to multiple imputation of
missing data, correct inferences from synthetic data require correct
pooling over the multiply imputed data sets.

Obtaining a final point estimate of the parameter of interest \(Q\)
after imputation is fairly easy and no different from pooling in the
case of missing data \citep{rubin_multiple_1987}. One can calculate the
average of the \(m\) point estimates \(q^{(l)}\) \[
\bar{q}_m = \sum_{l = 1}^m \frac{q^{(l)}}{m},
\] with \(l = 1, \dots, m\).

Also similarly to the missing data case, variances, and subsequently
confidence intervals, should incorporate the increase in variance that
is due to imputation
\citep{reiter_partially_inference_2003, drechsler_synthetic_2011}. Yet,
the increase in variance due to imputation differs according to whether
missing values are imputed or observed data is overimputed with missing
values. Whereas the variance estimate after imputation of missing data
needs to account for the fact that a certain amount of information in
the data is missing, variance estimation from synthetic data does not
suffer from this issue. The adjusted variance estimate that follows from
using multiple synthetic data sets only suffers from the fact that a
finite number of \(m\) synthetic data sets are used to resemble the
observed data. Hence, the according variance estimate for synthetic data
as developed by \citet{reiter_partially_inference_2003} yields \[
T = \bar{u}_m + \frac{b_m}{m},
\]

with between-imputation variance

\[
b_m = \sum_{l = 1}^m \frac{(q^{(l)} - \bar{q}_m)^2}{(m - 1)}, \\
\] and sampling variance

\[
\bar{u}_m = \sum_{l = 1} \frac{u^{(l)}}{m},
\] where \(u^{(l)}\) denotes the variance estimate in the \(l\)th
synthetic data set.

\hypertarget{results}{%
\section{Results}\label{results}}

We evaluate the synthetic data with respect to the \emph{true} data set
on the basis of three aims. We believe that every reliable and valid
data synthetisation effort in statistical data analysis should be able
to yield 1) unbiased univariate statistical properties, 2) unbiased
bivariate properties, 3) unbiased and valid multivariate inferences and
4) synthetic data that cannot be distinguished from real data. We
consider the evaluation of the synthetic data simulations in the above
order.

\hypertarget{univariate-estimates}{%
\subsection{Univariate estimates}\label{univariate-estimates}}

The univariate descriptives for the original data and the synthetic data
can be found in Table 2.

\begin{table}[ht]
\caption{Univariate descriptives for the true data and $m=5$ pooled univariate descriptives for the synthetic data over 1000 simulations. Variable names followed by a $^*$ are categorical.}
\centering
\begin{tabular}{rrrrrrrrr}
  \hline
 & n & mean & sd & median & min & max & skew & kurtosis \\ 
  \hline
original age & 748 & 9.16 & 6.89 & 10.50 & 0.04 & 21.18 & -0.03 & -1.56 \\ 
  synthetic age & 748 & 9.15 & 6.89 & 10.49 & 0.04 & 20.96 & -0.03 & -1.55 \\ 
  original hgt & 748 & 131.10 & 46.52 & 145.75 & 50.00 & 198.00 & -0.30 & -1.47 \\ 
  synthetic hgt & 748 & 131.06 & 46.50 & 145.32 & 50.69 & 197.16 & -0.30 & -1.47 \\ 
  original wgt & 748 & 37.12 & 26.03 & 34.55 & 3.14 & 117.40 & 0.38 & -1.03 \\ 
  synthetic wgt & 748 & 37.09 & 26.00 & 34.44 & 3.35 & 112.26 & 0.38 & -1.03 \\ 
  original bmi & 748 & 18.04 & 3.04 & 17.45 & 11.73 & 31.74 & 1.14 & 1.79 \\ 
  synthetic bmi & 748 & 18.05 & 3.08 & 17.48 & 11.49 & 32.37 & 1.11 & 1.85 \\ 
  original hc & 748 & 51.62 & 5.86 & 53.10 & 33.70 & 65.00 & -0.91 & 0.12 \\ 
  synthetic hc & 748 & 51.61 & 5.86 & 53.18 & 34.38 & 62.85 & -0.91 & 0.12 \\ 
  original gen* & 748 & 2.53 & 1.59 & 2.00 & 1.00 & 5.00 & 0.52 & -1.36 \\ 
  synthetic gen* & 748 & 2.53 & 1.59 & 2.00 & 1.00 & 5.00 & 0.52 & -1.35 \\ 
  original phb* & 748 & 2.75 & 1.86 & 2.00 & 1.00 & 6.00 & 0.56 & -1.25 \\ 
  synthetic phb* & 748 & 2.75 & 1.86 & 2.00 & 1.00 & 6.00 & 0.56 & -1.24 \\ 
  original tv & 748 & 8.43 & 8.12 & 3.00 & 1.00 & 25.00 & 0.85 & -0.78 \\ 
  synthetic tv & 748 & 8.42 & 8.11 & 3.19 & 1.00 & 25.00 & 0.85 & -0.77 \\ 
  original reg* & 748 & 3.02 & 1.14 & 3.00 & 1.00 & 5.00 & -0.08 & -0.77 \\ 
  synthetic reg* & 748 & 3.02 & 1.14 & 3.00 & 1.00 & 5.00 & -0.08 & -0.76 \\ 
   \hline
\end{tabular}
\end{table}

We see from Table 1 that the synthetic data estimates closely resemble
the true data estimates. All sample statistics of interest show
negligible bias over the 1000 synthetic data sets. Hence, univariately
the imputation model proves adequate.

\hypertarget{bivariate-estimates}{%
\subsection{Bivariate estimates}\label{bivariate-estimates}}

An often used bivariate statistic is Pearson's correlation coefficient.
When evaluating this correlation coefficient on the numeric columns in
the \texttt{boys} data set, we find that biases are very small. The
results are displayed in Table 3.

\begin{table}[H]
\caption{Bivariate correlations of the numerical columns in the true data with in parentheses the corresponding bias of the $m=5$ pooled synthetic correlations over 1000 simulations. All estimates are rounded to 3 decimal places. }
\centering
\begin{tabular}{rcccccc}
  \hline
 & age & hgt & wgt & bmi & hc & tv \\ 
  \hline
age & 1 & 0.976 (0.001) & 0.950 (0.000) & 0.627 (0.009) & 0.853 (0.000) & 0.810 (0.002) \\ 
  hgt & 0.976 (0.001) & 1 & 0.944 (0.001) & 0.596 (0.013) & 0.907 (0.000) & 0.754 (0.000) \\ 
  wgt & 0.950 (0.000) & 0.944 (0.001) & 1 & 0.791 (0.009) & 0.834 (0.000) & 0.817 (0.000) \\ 
  bmi & 0.627 (0.009) & 0.596 (0.013) & 0.791 (0.009) & 1 & 0.588 (0.009) & 0.610 (0.007) \\ 
  hc & 0.853 (0.000) & 0.907 (0.000) & 0.834 (0.000) & 0.588 (0.009) & 1 & 0.623 (0.000) \\ 
  tv & 0.810 (0.002) & 0.754 (0.000) & 0.817 (0.000) & 0.610 (0.007) & 0.623 (0.000) & 1 \\ 
   \hline
\end{tabular}
\end{table}

We see that the correlations obtained from synthetic data are unbiased
with respect to the true data set. The largest absolute bias over 1000
simulations equals 0.013, indicating that the imputation model is
capable of preserving the bivariate relations in the data.

\hypertarget{multivariate-model-inferences}{%
\subsection{Multivariate model
inferences}\label{multivariate-model-inferences}}

First, we evaluate the performance of our synthetic simulation set on a
linear model where \texttt{hgt} is modeled by a continuous predictor
\texttt{age} and an ordered categorical predictor \texttt{phb}. The
results for this simulation can be found in Table 4.

\begin{table}[ht]
\caption{Simulation results for a linear regression model with continuous and ordered categorical predictors. The model evaluated is \texttt{hgt $\sim$ age + phb}. Depicted are the true data estimate and the bias from the true data estimate and the coverage rate of the 95\% confidence interval for the bootstrap and synthetic data sets.}
\centering
\begin{tabular}{rrrrrr}
  \hline
  & & \multicolumn{2}{c}{Bootstrap} & \multicolumn{2}{c}{Synthetic}\\
 term & estimate & bias & cov & bias & cov \\ 
  \hline
(Intercept) & 63.087 & -0.001 & 0.970 & 0.405 & 0.958 \\ 
  age & 7.174 & 0.000 & 0.958 & -0.033 & 0.947 \\ 
  phb.L & -12.250 & 0.008 & 0.950 & 0.582 & 0.927 \\ 
  phb.Q & -1.376 & -0.022 & 0.926 & 0.112 & 0.934 \\ 
  phb.C & -3.564 & 0.051 & 0.915 & 0.301 & 0.912 \\ 
  phb\verb|^|4 & -0.431 & 0.016 & 0.930 & 0.106 & 0.940 \\ 
  phb\verb|^|5 & 2.064 & 0.060 & 0.941 & 0.077 & 0.943 \\ 
   \hline
\end{tabular}
\end{table}

We see that the finite nature of the true data set together with the
design-based simulation setup yields slight undercoverage for the dummy
variables of \texttt{phb}. This finding is observed in both the
bootstrap coverages (i.e.~the fraction of 95\% confidence intervals that
cover the true data parameters) and the synthetic data coverages. Hence,
it is likely that this undercoverage stems from the simulation setup,
rather than the imputation procedure. Besides the undercoverage, there
is a tiny bit of bias in the estimated coefficients of the variable
\texttt{phb} that occurs in the synthetic estimates, but not in the
observed estimates. Yet, since the bias is relatively small and does not
result in confidence invalidity, it seems fair to assume that the
introduced bias is not that problematic.

Second, we evaluate a proportional odds logistic regression model
wherein ordered categorical column \texttt{gen} is modeled by continuous
predictors \texttt{age} and \texttt{hc}, and categorical predictor
\texttt{reg}. The results for this model evaluation are shown in Table
5.

\begin{table}[H]
\caption{Simulation results for a proportional odds logistic regression model with continuous and ordered categorical predictors. The model evaluated is \texttt{gen $\sim$ age + hc + reg}. Depicted are the true data estimate and the bias from the true data estimate and the coverage rate of the 95\% confidence interval for the bootstrap and synthetic data sets.}
\centering
\begin{tabular}{rrrrrr}
  \hline
 & & \multicolumn{2}{c}{Bootstrap} & \multicolumn{2}{c}{Synthetic}\\
 term & estimate & bias & cov & bias & cov \\ 
  \hline
  age & 0.461 & 0.004 & 0.942 & 0.002 & 0.939 \\
  hc & -0.188 & -0.000 & 0.929 & -0.004 & 0.945 \\
  regeast & -0.339 & 0.012 & 0.960 & 0.092 & 0.957\\
  regwest & 0.486 & 0.009 & 0.952 & -0.122 & 0.944\\
  regsouth & 0.646 & 0.012 & 0.966 & -0.152 & 0.943 \\
  regcity & -0.069 & 0.012 & 0.940 & 0.001 & 0.972 \\
  G1$|$G2 & -6.322 & 0.032 & 0.934 & -0.254 & 0.946 \\
  G2$|$G3 & -4.501 & 0.052 & 0.936 & -0.246 & 0.945 \\
  G3$|$G4 & -3.842 & 0.058 & 0.937 & -0.244 & 0.948 \\
  G4$|$G5 & -2.639 & 0.064 & 0.936 & -0.253 & 0.947 \\
   \hline
\end{tabular}
\end{table}

These results demonstrate that the synthetic data analysis yields
inferences that are on par with inferences from the analyses directly on
the bootstrapped datasets. Hence, for the regression coefficients as
well as the intercepts, the analyses on the synthetic data yield valid
results. Nevertheless, a small amount of bias is introduced in the
estimated intercepts of the synthetic data. However, the corresponding
confidence interval coverage rates are actually somewhat higher than the
confidence interval coverage rates of the bootstrapped data. Therefore,
the corresponding inferences do not seem to be affected by this small
bias.

\hypertarget{data-discrimination}{%
\subsection{Data discrimination}\label{data-discrimination}}

When we combine the original and synthetic data, can we predict which
rows come from the synthetic data set? If so, then our synthetic data
procedure would be redundant, since the synthetic set differs from the
observed set. To evaluate whether we can distinguish between the true
data and the synthetic data, we combine the rows from each simulation
synthetic data set with the rows from the true data. We then run a
logistic regression model to predict group membership: i.e.~does a row
belong to the true data or synthetic data. As predictors we take all
columns in the data. The pooled parameter estimates over all simulations
can be found in Table 6.

\begin{table}[H]
\caption{Simulation results for a logistic regression model aimed at discriminating between synthetic records and true records.}
\centering
\begin{tabular}{rrrrrr}
  \hline
 term & estimate & std.error & statistic & df & p.value \\
  \hline
  (Intercept) & 0.22 & 1.15 & 0.19 & 521.93 & 0.60 \\ 
  wgt & 0.00 & 0.02 & 0.25 & 420.19 & 0.60 \\
  hgt & -0.00 & 0.01 & -0.18 & 359.73 & 0.60 \\
  age & -0.00 & 0.05 & -0.00 & 345.73 & 0.62 \\ 
  hc & 0.00 & 0.03 & 0.11 & 415.44 & 0.61\\
  gen.L & -0.00 & 0.42 & -0.00 & 164.76 & 0.65 \\
  gen.Q & -0.01 & 0.21 & -0.04 & 203.38 & 0.63 \\
  gen.C & -0.00 & 0.16 & -0.02 & 239.63 & 0.64 \\
  gen\verb|^|4 & 0.01 & 0.21 & 0.01 & 237.41 & 0.63 \\
  phb.L & -0.02 & 0.44 & -0.04 & 156.54 & 0.64 \\
  phb.Q & -0.01 & 0.22 & -0.02 & 198.20 & 0.64 \\
  phb.C & 0.00 & 0.18 & 0.00 & 211.17 & 0.62 \\
  phb\verb|^|4 & 0.00 & 0.18 & 0.02 & 228.51 & 0.64\\
  phb\verb|^|5 & 0.00 & 0.20 & 0.02 & 248.57 & 0.63\\
  tv & 0.00 & 0.02 & 0.01 & 264.26 & 0.63\\
  regeast & -0.00 & 0.23 & 0.01 & 210.90 & 0.64 \\ 
  regwest & -0.00 & 0.22 & -0.00 & 221.14 & 0.63 \\ 
  regsouth & -0.01 & 0.22 & -0.02 & 226.10 & 0.64 \\ 
  regcity & 0.01 & 0.27 & 0.03 & 204.15 & 0.64 \\ 
  bmi & -0.02 & 0.06 & -0.26 & 320.18 & 0.61 \\
   \hline
\end{tabular}
\end{table}

From these pooled results we can see that the effects for all predictors
are close to zero and non-significant. When we take the predicted values
from the simulated models and compare them with the \emph{real} values,
we obtain the summary statistics in Table 7.

\begin{table}[H]
\caption{Confusion statistics for a prediction model aimed at discriminating between synthetic records and true records.}
\centering
\begin{tabular}{rr}
  \hline
& estimate\\
  \hline
 Accuracy & 0.50381 \\
 Balanced Accuracy & 0.50381 \\
 Kappa & 0.00762 \\
 McnemarPValue & 0.63187 \\
 Sensitivity & 0.50368 \\
 Specificity & 0.50394 \\
 Prevalence & 0.50000 \\
   \hline
\end{tabular}
\end{table}

The accuracy of the predictive modeling effort is not better than random
selection and the Kappa coefficient indicates that a perfect prediction
model is far from realized. The accuracy is quite balanced as there is
no skewness over sensitivity and specificity. These findings indicate
that the synthetic data is indistinguishable from the true data.

\hypertarget{discussion}{%
\section{Discussion}\label{discussion}}

We demonstrate that generating synthetic data sets with \texttt{mice} in
\texttt{R} is a straightforward process that fits well in a data
analysis pipeline. The approach is hardly different from using multiple
imputation to solve problems related to missing data, and hence can be
expected to be familiar to applied researchers. The multiple synthetic
sets yield valid inferences on the true underlying data generating
mechanism, thereby capturing the nature of the original data. This makes
the multiple synthetisation procedure with \texttt{mice} suitable for
further dissemination of synthetic data sets.

To some, the procedure of generating multiple synthetic sets may seem
overly complicated. We would like to emphasize that analyzing a single
synthesized set, while perhaps unbiased, would underestimate the
variance properties that are so important in drawing statistical
inferences from finite data sets. After all, we are often not interested
in the sample at hand, but aim to make inferences about the underlying
data generating mechanism as reflected in the population. Properly
capturing the uncertainty of synthetic data sets, just like with
incomplete data sets, is therefore paramount.

Besides capturing uncertainty of synthetic data, it is important that
imputers pay close attention to disclosure risks that remain after
creating synthetic data sets. Creating synthetic data, unless generated
from a completely parametric distribution, does not remove all potential
disclosure risks. For example, if the values that ought to be replaced
get the exact same value imputed, the synthetisation procedure has no
use.

Additionally, if not all variables in the data are synthesized, but the
variables that are synthesized can be linked to open access data, a
disclosure risk may remain \citep{drechsler2011empirical}. If the open
access data allows for identification, and the corresponding
observations in the synthetic data can be identified, the variables that
are not synthesized may provide information that should not have been
disseminated. The associated problems generally decrease when the
sensitive variables are synthesized as well. Still, it is important to
remain critical regarding the extent to which the synthetic data might
be disclosive. The practical development of easy to use software to
identify which observations are at risk of disclosure is an area of
future work that can improve these issues. Subsequently, the
implementation of ways to overcome such problems, for example by record
swapping as suggested by \citet{jiang2021balancing}, once detected is
welcomed.

That said, while the fields of data disclosure control and data
synthetisation originated decades ago, it is more relevant today than
ever. In our study we outline a standard research workflow where the
goal is to synthesize a complete data set. To simplify the simulation we
adopted a bootstrapping scheme to induce sampling variance in order to
conform to the combination procedure defined by
\citet{reiter_partially_inference_2003}. Ideally, one would like to omit
the bootstrap from the synthetization to adopt a scheme where the
sampled data itself serves as a reference: much like the procedure
outlined in \citet{vink2014pooling} for incomplete data simulation. The
corresponding pooling rules have not been derived yet and the
incorporation in data analysis workflows would require proper attention
from developers alike.

It is important to note that in our simulations we used a single
iteration. A single iteration is sufficient only when the true data is
completely observed, or when the missingness pattern is monotone
\citep{drechsler_synthetic_2011}. If both observed and unobserved values
are to be synthesized, then more iterations and a careful investigation
into the convergence of the algorithm are in order. Synthetic data
generation with \texttt{mice} is therein no different than multiple
imputation with \texttt{mice}.

There are other developments that can generate synthetic data sets. For
example, the \texttt{synthpop} \citep{synthpop} package in \texttt{R}
would yield valid synthetic data sets, but requires the true data to be
completely observed. To avoid this, one could adopt a two-step approach
wherein the incomplete values are multiply imputed before
synthetization. Given \(m\) multiple imputations and \(r\)
synthetizations, at least \(m \times r\) synthetic data sets are then in
order. However, this approach requires that researchers first impute the
missingness using a different package, and then use \texttt{synthpop} to
create synthetic data sets, while \texttt{mice} allows to do both.
Additionally, the flexibility with \texttt{mice} is that both unobserved
and observed data values could be synthesized at once, without the need
for a two-step approach. Then, using \texttt{mice}, \(m\) synthetic sets
are sufficient. As of today, no pooling rules for one-step imputation of
missingness and synthetisation have been developed, but the derivation
of those would further reduce the burden of creating synthetic data
sets.

The ease of use when synthesizing data with \texttt{mice} in \texttt{R},
together with the validity of inferences obtained through this procedure
opens up a wealth of possibilities for data dissemination and further
research of initially private data.

% %%%%%%%%%%%%%%%%%%%%%%%%%%%%%%%%%%%%%%%%%%
% %% optional
% \supplementary{The following are available online at www.mdpi.com/link, Figure S1: title, Table S1: title, Video S1: title.}
%
% % Only for the journal Methods and Protocols:
% % If you wish to submit a video article, please do so with any other supplementary material.
% % \supplementary{The following are available at www.mdpi.com/link: Figure S1: title, Table S1: title, Video S1: title. A supporting video article is available at doi: link.}

\vspace{6pt}

%%%%%%%%%%%%%%%%%%%%%%%%%%%%%%%%%%%%%%%%%%
\acknowledgments{We are grateful to Mirthe Hendriks and Stijn van den
Broek for replicating the simulation study on an independent data set.}

%%%%%%%%%%%%%%%%%%%%%%%%%%%%%%%%%%%%%%%%%%
\authorcontributions{The authors contributed equally to this work.}

%%%%%%%%%%%%%%%%%%%%%%%%%%%%%%%%%%%%%%%%%%
\conflictsofinterest{The authors declare no conflict of interest.}

%%%%%%%%%%%%%%%%%%%%%%%%%%%%%%%%%%%%%%%%%%
%% optional

\input{"appendix.tex"}

%%%%%%%%%%%%%%%%%%%%%%%%%%%%%%%%%%%%%%%%%%
% Citations and References in Supplementary files are permitted provided that they also appear in the reference list here.

%=====================================
% References, variant A: internal bibliography
%=====================================
%\reftitle{References}
%\begin{thebibliography}{999}
% Reference 1
%\bibitem[Author1(year)]{ref-journal}
%Author1, T. The title of the cited article. {\em Journal Abbreviation} {\bf 2008}, {\em 10}, 142--149.
% Reference 2
%\bibitem[Author2(year)]{ref-book}
%Author2, L. The title of the cited contribution. In {\em The Book Title}; Editor1, F., Editor2, A., Eds.; Publishing House: City, Country, 2007; pp. 32--58.
%\end{thebibliography}

% The following MDPI journals use author-date citation: Arts, Econometrics, Economies, Genealogy, Humanities, IJFS, JRFM, Laws, Religions, Risks, Social Sciences. For those journals, please follow the formatting guidelines on http://www.mdpi.com/authors/references
% To cite two works by the same author: \citeauthor{ref-journal-1a} (\citeyear{ref-journal-1a}, \citeyear{ref-journal-1b}). This produces: Whittaker (1967, 1975)
% To cite two works by the same author with specific pages: \citeauthor{ref-journal-3a} (\citeyear{ref-journal-3a}, p. 328; \citeyear{ref-journal-3b}, p.475). This produces: Wong (1999, p. 328; 2000, p. 475)

%=====================================
% References, variant B: external bibliography
%=====================================
\reftitle{References}
\externalbibliography{yes}
\bibliography{synth.bib}

%%%%%%%%%%%%%%%%%%%%%%%%%%%%%%%%%%%%%%%%%%
%% optional
\sampleavailability{A full simulation archive is available from
\url{https://github.com/amices/Federated_imputation/tree/master/manuscript}.}

%% for journal Sci
%\reviewreports{\\
%Reviewer 1 comments and authors’ response\\
%Reviewer 2 comments and authors’ response\\
%Reviewer 3 comments and authors’ response
%}

%%%%%%%%%%%%%%%%%%%%%%%%%%%%%%%%%%%%%%%%%%
\end{document}
